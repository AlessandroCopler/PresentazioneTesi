\documentclass{beamer}
\usepackage[italian]{babel}
%\usepackage[latin1]{inputenc}
\usepackage[T1]{fontenc}
%%%%%%%%%%%%%%%%%%%%%%%%%
%\usepackage{color}
%\usepackage{amsmath}
%\usepackage{minted}
%\usepackage[italian]{babel}
%\usepackage{graphics}
%\usepackage{graphicx}
%\usepackage[subfigure]{}
%\usepackage{setspace}
\usepackage[utf8]{inputenc}%
%%%%%%%%%%%%%%%%%%%%%%%%%





\title{Metodologie e tecniche di Continuous Integration per l'evoluzione di OpenStack}

%\author[Euclide]{Euclide di Alessandria \\
%\texttt{euclide@alessandria.edu}}
%\date[VII SINP]{VII Simposio Internazionale sui Numeri Primi}

\author[A.Copler e L.Calomeni]{
  Alessandro Copler\\
  \texttt{alessandro.copler@gmail.com}\\
  \and
  Luca Calomeni\\
  \texttt{lcalomeni@gmail.com}
}


\institute[UniDiBergamo]{Università degli studi di Bergamo - Facolta di Ingegneria}
%\logo{\includegraphics[width=3cm]{unibg.jpg}}

\usetheme{Hannover}
%\usecolortheme{spruce}
%\usetheme{AnnArbor}
%\useoutertheme[right]{sidebar}
\setbeamercovered{dynamic}
\theoremstyle{definition}
\newtheorem{definizione}{Definizione}
\theoremstyle{plain}
\newtheorem{teorema}{Teorema}
\begin{document}
\begin{frame}
\maketitle
\end{frame}


\begin{frame}
\frametitle{Piano della presentazione}
\tableofcontents
\end{frame}

\section{Introduzione}

\subsection{CLoud
Computing}
\begin{frame}
\frametitle{Che cos'è il Cloud Computing?}
\begin{definizione}
Con il termine \alert{Cloud Computing} si intende un insieme di tecnologie informatiche che.

\begin{figure}[!h]
	\begin{center}
\includegraphics[width=3cm]{cloud.png}
\end{center}
\end{figure}

\end{definizione}
\end{frame}
%%%%%%%%%%%%%%%%%%%%%%%%%%%%%%%%%%%%%%%%%%%%%%%%%%%%%%%%%%%%%%%%%%%%%%5
\subsection{OpenStack}
\begin{frame}
\frametitle{Che cos'è OpenStack?}
\begin{definizione}
\alert{OpenStack} bla bla bla

\begin{figure}[!h]
	\begin{center}
\includegraphics[width=3cm]{openstack.png}
\end{center}
\end{figure}

\end{definizione}
\end{frame}
%%%%%%%%%%%%%%%%%%%%%%%%%%%%%%%%%%%%%%%%%%%%%%%%%%%%%%%%%%%%%%%%%%%%%%%%%%%%
\subsection{Continuous Integration con Jenkins}
\begin{frame}
\frametitle{Cosa si intende per Continuous Integration?}
\begin{definizione}
Per \alert{Continuous Integration} si intende  bla bla bla
\end{definizione}
\end{frame}


\begin{frame}
\frametitle{Jenkins per la Continuous Integration}
\begin{definizione}
\alert{Jenkins} bla bla bla
\end{definizione}
\end{frame}
%%%%%%%%%%%%%%%%%%%%%%%%%%%%%%%%%%%%%%%%%%%%%%%%%%%%%%%%%%%%%%%%%%%%%%%%%%%%

\section{Stato dell'arte}
\subsection{Processo di validazione della Fondazione OpenStack}
\begin{frame}
\frametitle{Come si contribuisce OpenStack}
\begin{definizione}
 \alert{contribuire} 
\end{definizione}
\end{frame}

\section{Configurazioni studiate}
\subsection{Prima configurazione con singolo progetto}
\begin{frame}
\frametitle{Configurazione singolo progetto}
\begin{definizione}
Un \alert{singolo progetto}
\end{definizione}
\end{frame}

\subsection{Seconda configurazione con multi progetti}
\begin{frame}
\frametitle{Configurazione multi progetto}
\begin{definizione}
Un \alert{clonazione full} 
\end{definizione}
\begin{definizione}
Un \alert{clonazione linked} 
\end{definizione}
\end{frame}

\subsection{Terza configurazione multi progetti indipendenti}
\begin{frame}
\frametitle{Configurazione multi progetti indipendenti}
\begin{definizione}
Un \alert{multi progetti indipendenti} è un intero $>1$ che ha esattamente
due divisori positivi.
\end{definizione}
\end{frame}

\section{Benchmark dei risultati}

\subsection{Benchmark}
\begin{frame}
\frametitle{Benchmark dei tempi medi ottenuti}
\begin{definizione}
 \alert{benchmark} bla bla
\end{definizione}
\end{frame}

\subsection{Trade-off}
\begin{frame}
\frametitle{Trade-off performance}
\begin{definizione}
\alert{trade-off} bla bla
\end{definizione}
\end{frame}

\section{Considerazioni finali e possibili sviluppi futuri}
\subsection{Commenti finali}
\begin{frame}
\frametitle{Che cosa sono i numeri primi?}
\begin{definizione}
Un \alert{numero primo} è un intero $>1$ che ha esattamente
due divisori positivi.
\end{definizione}
\end{frame}
\subsection{Sviluppi futuri}
\subsection{Ringraziamenti}
\end{document}